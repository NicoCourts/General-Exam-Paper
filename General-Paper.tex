\documentclass[12pt]{article}

\usepackage{setspace}

\usepackage{graphicx, color, fancyhdr, tikz-cd, enumitem, framed, adjustbox, bbm, upgreek, xcolor, manfnt}
\usepackage[framed,thmmarks]{ntheorem}
\usepackage[framemethod=tikz]{mdframed}
\usepackage{hyperref}
\hypersetup{
	colorlinks = true,
	linkcolor = [rgb]{0,0,0.5},
	citecolor = [rgb]{0.6,0,0},
	urlcolor = [rgb]{0,0,0.5}
}
\usepackage[style=alphabetic, bibencoding=utf8]{biblatex}
%Set the bibliography file
\bibliography{sources}

\usepackage[T1]{fontenc}
\usepackage[urw-garamond]{mathdesign}
\usepackage{garamondx}
\usepackage{ytableau}

%Replacement for the old geometry package
\usepackage{fullpage}
\usepackage{amsmath}

%Input my definitions
/home/nico/latex-includes/mydefs.tex

%Shade definitions
\theoremindent0cm
\theoremheaderfont{\normalfont\bfseries} 
\def\theoremframecommand{\colorbox[rgb]{0.9,1,.8}}
\newshadedtheorem{defn}[thm]{Definition}

%Set apart my theorems and lemmas and such
\surroundwithmdframed[outerlinewidth=0.4pt,
  innerlinewidth=0.5pt,
  middlelinewidth=1pt,
  middlelinecolor=white,
  topline=false,bottomline=false,rightline=false,leftmargin=2em]{thm}
\surroundwithmdframed[outerlinewidth=0.4pt,
  innerlinewidth=0.5pt,
  middlelinewidth=1pt,
  middlelinecolor=white,
  topline=false,bottomline=false,rightline=false,leftmargin=2em]{lem}
\surroundwithmdframed[outerlinewidth=0.4pt,
  innerlinewidth=0.5pt,
  middlelinewidth=1pt,
  middlelinecolor=white,
  topline=false,bottomline=false,rightline=false,leftmargin=2em]{cor}
  \surroundwithmdframed[outerlinewidth=0.4pt,
  innerlinewidth=0.5pt,
  middlelinewidth=1pt,
  middlelinecolor=white,
  topline=false,bottomline=false,rightline=false,leftmargin=2em]{prop}

%%%%%%%%%%%%%%%%%%%%%%%%%%%%%%%%%%%%%%%%%%%%%%%%%%%%%%%%%%%%%%%%%%%%%%
%%%%%%%%%%%%%%%%%%%%%%% Customize Below %%%%%%%%%%%%%%%%%%%%%%%%%%%%%%
%%%%%%%%%%%%%%%%%%%%%%%%%%%%%%%%%%%%%%%%%%%%%%%%%%%%%%%%%%%%%%%%%%%%%%

%header stuff
\setlength{\headsep}{24pt}  % space between header and text
\pagestyle{fancy}     % set pagestyle for document
\lhead{General Exam Paper} % put text in header (left side)
\rhead{Nico Courts} % put text in header (right side)
\cfoot{\itshape p. \thepage}
\setlength{\headheight}{15pt}
%\allowdisplaybreaks

% Document-Specific Macros


\begin{document}
%make the title page
\title{General Exam Paper \vspace{-1ex}}
\author{Nico Courts}
\date{Winter 2019}
\maketitle

\begin{abstract}
	We begin by going through a considerable amount of domain knowledge concerning representations of $\GL_n$,
	representations of $\frakS_n$, and strict polynomial functors all in service of understanding the Schur-Weyl 
	functor that relates several of these categories. From there, we investigate recent work on the part of Krause 
	and his students Aquilino and Reischuk on this functor and the fact that it is monoidal under reasonably natural monoidal structures on 
	the categories in question. Finally we ask some questions about whether the monoidal structure on strict polynomial functors 
	extends meaningfully to pathologies that arise in positive characteristic.
\end{abstract}

\newpage
\renewcommand{\baselinestretch}{0.75}\normalsize
\setcounter{tocdepth}{3}
\tableofcontents
\renewcommand{\baselinestretch}{1.0}\normalsize

\newpage
\section{Introduction}
\subsection{Schur and Polynomial Representations}
The story of this project (more-or-less) begins with Schur's doctoral thesis \cite{schur-thesis} in which he defined
polynomial representations of $\GL_n$---a theory which he developed more completely in his later paper \textit{\"Uber die 
rationalen Darstellungen der allgemeinen linearen Gruppe}\footnote{English: \textit{On the rational representations of the general linear group}}
\cite{schur-rational}. In these papers, Schur develops the idea of a \textbf{polynomial representation of $\GL_n$},
meaning a (finite dimensional) representation where the coefficient functions of the representing map 
\[\rho:\GL_n\to \GL(V)\]
is polynomial in each coordinate. For example, the map sending 
\[A=\begin{pmatrix}
	a&b\\
	c&d
\end{pmatrix}\mapsto \begin{pmatrix}
	a^2d-abc & acd-c^2b & 0\\
	abd-b^2c & ad^2-bcd & 0\\
	0 & 0 & ad-bc
\end{pmatrix}=\rho(A)\]
is a three-dimensional polynomial representation of $\GL_2$.

The block-diagonal form above demonstrates a direct sum decomposition of our representation into two parts: one two-dimensional homogeneous degree 3
and one one-dimensional homogeneous degree 2 (in the entries of $A$). A result in \cite{schur-thesis} tells us that, in fact, this can always be done: 
if $V$ is a polynomial representation of $\GL_n$,
then $V$ decomposes as a direct sum of representations 
\[V=\bigoplus_\delta V_\delta\]
where each $V_\delta$ is a polynomial representation where the coefficient functions are \textit{homogeneous degree $\delta$}. 
This allows us to focus our attention to the structure of these $V_\delta$ as the fundamental building blocks of the theory.

The key insight made in this theory comes from the observation that the vector space (recall $V\cong k^n$)
\[E=V^{\otimes r}\]
is made into a $(\GL_n(k),\frakS_r)$-bimodule in a very natural way, and that this bimodule gives us a way to relate 
$\rmod {\frakS_r}$ with (a subcategory of) $\lmod {\GL_n(k)}$ via the so-called \textbf{Schur-Weyl functor.}

\subsection{The Schur-Weyl Functor}
Clearly a connection between representations of two groups that are so ubiquitous in group theory and math in general 
is a stunning observation, and much effort has been expended since the late 20th century to study this functor and its 
properties---especially in how it relates the representation theory of these two groups. 

For instance, Friedlander and Suslin \cite{friedlander-suslin}
originally discussed the idea of \textbf{strict polynomial functors} and showed that the category of repesentations 
of the Schur algebra $S(n,d)$ was equivalent to the category $\calP_d$ of homogeneous degree $d$ strict polynomial functors.

In later work, Krause \cite{krause-strict-poly-func} used an alternative construction of $\calP_d$ as the category of
of reprsentations of the $d$-divided powers of the category of finitely generated projective $k$-modules. The upshot being that 
the latter object $\Gamma^d P_k$ has an obvious monoidal structure which $\calP_d$ inherits in a natural way. This new concrete 
monoidal structure opens up the field to discussing several notions of duality defined in different contexts 
and solidifying connections between them.

Krause's students Aquilino and Reischuk, in their paper \cite{aquilino-reischuk}, prove, among other facts, that 
under these natural monoidal structures the Schur-Weyl functor is in fact monoidal. This puts the theory of representations 
of these groups and algebras firmly in the realm of monoidal categories, opening up the area to new questions using 
tools from category theory.

\subsection{Notation and Conventions}\label{subsec:notation}
Throughout this paper we will define $k$ to be a field (not necessarily of characteristic zero or algebraically closed unless otherwise noted).
%Let $\Alg_k$ be the category of $k$-algebras and $\Grp$ denote the category of groups with homorphisms.

We will use $\Gamma=\Gamma_k=\GL_n=\GL_n(k)$ to denote the general linear group, $\Aut_k(k^n)$. Let $\frakS_r$ denote the symmetric group on $r$ letters.

When speaking of the ($k$) vector space spanned by elements $v_1,\dots,v_n$, we will use the notation 
\[\langle v_1,\dots, v_n\rangle.\]

%%%%%%%%%%%%%%%%%%%%%%%%%%%%%%%%%%%%%%%%%%%%%%%%%
%%%%%%%%%%%%%%%%%%%%%%%%%%%%%%%%%%%%%%%%%%%%%%%%%
%%%%%%%%%%%%%%%%%%%%%%%%%%%%%%%%%%%%%%%%%%%%%%%%%
%%%%%%%%%%%%%%%%%%%%%%%%%%%%%%%%%%%%%%%%%%%%%%%%%
%%%%%%%%%%%%%%%%%%%%%%%%%%%%%%%%%%%%%%%%%%%%%%%%%

\newpage
\section{Representations of \texorpdfstring{$\Gamma$}{Gamma} and of \texorpdfstring{$\frakS_n$}{Sn}}
We begin by detailing the theory behind the (polynomial) representations of $\GL_n$ as well as the representations of $\frakS_n$ to 
familiarize ourselves with the classical representation theory associated to these groups.

\subsection{Polynomial representations of \texorpdfstring{$\Gamma$}{Gamma}}
Let $\Gamma$ be the affine group scheme defined in section~\ref{subsec:notation}. Then 
\begin{defn}
	A (finite dimensional) \textbf{representation} of $\Gamma$ is a (finite dimensional) vector space $V$ along with a map
	\[\rho:\Gamma\to \GL(V)\eqdef\Aut_k(V)\]
	where $\rho$ is a group homomorphism.
\end{defn}
\begin{rmk}
	Often in representation theory one combines the map $\rho$ and vector space $V$ into a single object:
	a $k\Gamma$-module. This simultaneously encodes the vector space structure (via $k$-linearity) and the action by 
	$\Gamma$.
\end{rmk}
Representations of $\Gamma$ can be, in general, ``analytic.'' One can check that the map 
\[\rho:k^\times=\GL_1(k)\to \GL(k^2)\qquad\text{via}\qquad x\mapsto\begin{pmatrix}
	1 & \ln |x|\\ 0 & 1
\end{pmatrix}\]
gives a group homomorphism (and thus representation) between these two groups, but the logarithm makes this representation decidedly \textit{not algebraic.}
To narrow our focus somewhat and ensure we stay within the realm of algebra, we make the following definition:
\begin{defn}\label{def:poly-rep}
	A \textbf{polynomial representation} of $\Gamma$ is a representation $\rho$ such that the structure maps of $\rho$ 
	are polynomials in the functions $c_{ij}:\Gamma\to k$ that extract the $(i,j)^{th}$ entry.
\end{defn}
\begin{rmk}
	Recall (or learn for the first time!) that the \textit{structure maps} of a representation $(\rho,V)$ are a collection 
	of maps $r_{ij}$ for $1\le i,j,\le n$ from $\Gamma$ to $k$ such that for all $g\in \Gamma$:
	\[g\cdot v_i=\sum_{j=1}^n r_{ij}(g)v_j\]
	where we have picked a basis $\{v_1,\dots,v_n\}$ for $V$. Of course changing basis may change our 
	$r_{ij}$, but an invariant of the representation is the \textbf{span} $\langle r_{ij}\rangle$.
\end{rmk}
\begin{rmk}
	If all $r_{ij}$ are homogeneous polynomials of the same degree, we say that $\rho$ is a \textbf{homogeneous polynomial representation} of $\Gamma$.
\end{rmk}
\begin{defn}\label{def:Mnr}
	Let $M_k(n)=M(n)$ be the collection of all polynomial representations of $\GL_n$ and let $M_k(n,r)=M(n,r)$ 
	be the collection of all degree $r$ polynomial representations of $\GL_n$.

	Generally speaking, we will identify both of these with an appropriate subcategory of $\lmod{k\Gamma}$.
\end{defn}
It is the \textit{polynomial} representations that we will concern ourselves with in the following sections. 

\subsubsection{Reducing scope}
Using some of our familiar friends from representation theory (as well as some clever twists), 
we can simplify this picture considerably by proving the following structural result:
\begin{thm}[{\cite[pp.7-10]{schur-thesis}}]\label{thm:decomp}
	Every polynomial representation $V$ over an infinite field $k$ decomposes as a direct sum 
	\[V\cong\bigoplus_{\delta\in\bbN}V_\delta\]
	where $V_\delta$ is a \textit{homogeneous} polynomial representation of degree $\delta.$
\end{thm}
Clearly, then, it suffices to understand the \textit{homogeneous degree $r$} polynomial representations of $\Gamma$ if we are looking
to understand the larger structure.

\brk

We begin with a useful lemma extracted from a proof in \cite{schur-thesis} echoing the general theory of 
orthogonal decomposition of Artinian algebras.
\begin{lem}\label{lem:orth-decomp}
	Let $C_0,\dots,C_m\in M_n(k)$ be mutually orthogonal idempotent matrices that sum to the identity. That is, 
	\[I_n=\sum_i C_i\quad\text{and}\quad C_iC_j=\delta_{ij}C_i\]
	for all $0\le i,j\le m$. Then there exists an invertible matrix $P$ such that for some positive integers $d_0,\dots,d_m$ with $\sum_k d_k=n$ and for all $i$,
	\[P^{-1}C_iP=\begin{pmatrix}
		\mathbf{0}_{N_i} & &\\
		& I_{d_i} & \\
		& & \mathbf{0}_{M_i}
	\end{pmatrix}\]
	Where $N_i=\sum_{0\le j<i}d_j$ and $M_i=n-d_i-N_i$
\end{lem}
\begin{prf}[of lem~\ref{lem:orth-decomp}]
	We set $S_k=\{C_0,C_1,\dots,C_k\}$ and we proceed by induction on $k$. When $k=0$, $S_k=\{C_0\}$. Now since 
	$C_0^2=C_0$, we get that 1 and 0 are the only eigenvalues of $C_0$, so there is an $r\times r$ matrix $P_0$ 
	and a positive integer $d_0$ such that
	\[P_0^{-1}C_0P_0=\begin{pmatrix}
		I_{d_0} & \\
			& \mathbf{0}_{n-d_0}
	\end{pmatrix}.\]
	which establishes the base case.

	Now assume that we have a matrix $P_{k-1}$ such that this property holds for all elements of $S_{k-1}$.
	Define, for each $0\le i\le k$, 
	\[C_i'\eqdef P^{-1}_{k-1}C_iP_{k-1}\]
	and since the $C_k$ is assumed to be orthogonal to all other $C_i$,
	\[C_k'=\begin{pmatrix}
		\mathbf{0}_{N_k} & \\
		& D_k
	\end{pmatrix}\]
	for some $D_k$.

	Now by properties of block diagonal matrices, we have 
	\[D_k^2=D_k\]
	so the eigenvalues of $D_k$ are again one and zero. Thus there is an invertible $Q\in \GL_{n-N_k}$ such that 
	\[Q^{-1}D_kQ=\begin{pmatrix}I_{d_k} &\\ & \mathbf{0}_{M_k}\end{pmatrix}\]
	and so by setting
	\[P_k\eqdef P_{k-1}\begin{pmatrix}I_{N_k} &\\ & Q\end{pmatrix}\]
	we can define
	\[C''_i\eqdef P_k^{-1}C_i P_k=\begin{pmatrix}I_{N_k} &\\ & Q\end{pmatrix}^{-1}C'\begin{pmatrix}I_{N_k} &\\ & Q\end{pmatrix}\]
	for $0\le i\le k$, we see immediately that $C_i'=C_i''$ for $0\le i<k$ and furthermore 
	\[C_k''=\begin{pmatrix}
		\mathbf{0}_{N_k} & \\
		& Q^{-1}D_kQ
	\end{pmatrix}=\begin{pmatrix}
		\mathbf{0}_{N_k} & &\\
		& I_{d_k} & \\
		& & \mathbf{0}_{M_k}
	\end{pmatrix}\]
	completing the inductive step. This this result holds for all $S_i$ and in particular for $S_m$, so the result is proven.
\end{prf}
As well as another result on a special class of commuting block diagonal matrices:
\begin{lem}\label{lem:block-diag}
	Let $k$ be an infinite field and let $A$ be a block diagonal matrix over $k$ of the form
	\[A=\operatorname{diag}(x^mI_{d_m},x^{m-1}I_{d_{m-1}},\dots,I_{d_0})\]
	where $d_i$ is (clearly) the dimension of the $(m-i)^{th}$ block and let $B$ be any matrix that commutes with $A$
	for every choice of $x\in k$. Then $B$ is block diagonal of the same shape as $A$. 
\end{lem}
\begin{prf}[of lem~\ref{lem:block-diag}]
	We proceed by comparing the entries in $AB$ and $BA$: notice that 
	\[(AB)_{ij}=\sum_k A_{ik}B_{kj}=A_{ii}B_{ij}=x^aB_{ij}\]
	and 
	\[(BA)_{ij}=\sum_k B_{ik}A_{kj}=B_{ij}A_{jj}=x^bB_{ij}.\]
	We will show that if the $(i,j)^{th}$ postion is not in one of the blocks of $A$, then it is zero.

	But if $(i,j)$ is not in one of the blocks of $A$, then the nonzero element in the $i^{th}$ row and the nonzero element in 
	the $j^{th}$ column ($x^a$ and $x^b$ in the above equations) are not the same! Since $x$ is arbitrary, this forces $B_{ij}=0$,
	so $B$ is block diagonal with blocks the same as $A$.
\end{prf}
\begin{rmk}
	Notice that in the above proof we used implicitly that there is an $x\in k$ such that for all $a,b$
	\[x^a=x^b\quad\Rightarrow\quad a=b\]
	which is true if (and only if!) $k$ is infinite. This is because any such $x$ must be a root of $x^n=x$ for some $n$, which 
	has finitely many roots over any field (but every element in $\bbF_p$ satisfies $x^p=x$).
\end{rmk}

And finally using these two lemmas allows us to prove our main result:
\begin{prf}[of thm~\ref{thm:decomp}]
	We recreate the argument in Schur's thesis, translated from German and reinterpreted in more modern parlance. 
	
	Let $(\rho,V)$ be a polynomial representation of $\Gamma$ with $\dim_k V=r$. Then let $x\in k^\times$ be arbitrary (thought of as an indeterminate)
	and consider the matrix $xI_n\in\Gamma$. The image of this matrix under $\rho$ is a matrix 
	\[\rho(A)=\begin{pmatrix}
		p_{11}(x) & \cdots & p_{1r}(x)\\
		\vdots & \ddots & \vdots\\
		p_{r1}(x) & \cdots & p_{rr}(x)
	\end{pmatrix}\]
	where each $p_{ij}$ is a polynomial in $x$. Let $m=\max_{i,j}\deg p_{ij}$, and this gives us a decomposition 
	\[\rho(A)=x^m C_0+x^{m-1}C_1+\cdots+ xC_{m-1}+C_m\]
	where each $C_i$ is an $r\times r$ matrix.

	Let $y$ be another indeterminate and $B=yI_n$. By virtue of being a representation of $\Gamma$, we get 
	\[\rho(A)\rho(B)=\rho(xI_n)\rho(yI_n)=\rho(xyI_n)=\rho(AB)\]
	and using this setup we prove the following result: 
	\[\text{For all $0\le i,j\le m$, with the $C_l$ as above,}\quad C_iC_j=\delta_{ij}C_i\]
	That this is true can be established by comparing coefficients in the equation
	\begin{align*}
		\rho(AB)&=\rho(A)\rho(B)\\
		C_0(xy)^m+\cdots+C_i(xy)^{m-i}+\cdots+C_m&=C_0^2x^my^m+\cdots+C_iC_jx^{m-i}y^{m-j}+\cdots+C_m^2
	\end{align*}
	Indeed, we immediately get that $C_i=C_i^2$ and furthermore the coefficients on $x^iy^j$ when $i\ne j$ give us
	\[0=C_{m-i}C_{m-j}.\]
	
	Thus we have shown that the $C_i$ form a set of orthogonal idempotent matrices and evaluating our original equation at $x=1$,
	we get (since $\rho$ is a homomorphism)
	\[I_r=1C_0+\cdots+1C_m=\sum C_i\]
	so the result from lemma~\ref{lem:orth-decomp} applies: we get a matrix $P$ such that 
	\[P^{-1}\rho(xI_n)P=\begin{pmatrix}
		x^mI_{d_0} & & & &\\
		& x^{m-1}I_{d_1} & & &\\
		& & \ddots & &\\
		& & & xI_{d_{m-1}} & \\
		& & & & I_{d_m}
	\end{pmatrix}\]
	Now let $\rho'(g)=P^{-1}\rho (g)P$ for all $g\in\Gamma$. This is a representation of $\Gamma$ since it it differs from $\rho$ by 
	an automorphism of $\GL(V)$. Since matrix multiplication is an algebraic operation, $\rho'$ is still a polynomial representation of $\Gamma$. 
	But notice that for all $g\in\Gamma$
	\[\rho'(g)\rho'(xI_n)=\rho'(xg)=\rho'(xI_n)\rho'(g)\]
	Then lemma \ref{lem:orth-decomp} gives us that $\rho'(g)$ decomposes in the same way for all $g\in \Gamma$, so 
	we know that $\rho'$ decomposes as a direct sum of representations 
	\[\rho'=\sum_{i=0}^m \rho'_i\]
	where for each $i$ and $\lambda\in k$,
	\[\rho'_i(\lambda g)=\rho_i'(\lambda I_{d_i})\rho_i'(g)=\lambda^i\rho'_i(g)\]
	so each $\rho_i'$ is a homogeneous degree $i$ polynomial representation of $\Gamma$.

	But of course the decomposition of a representation is independent of choice of basis,
	so we get a decomposition of $\rho$ into homogenous pieces, as desired.
\end{prf}

\subsubsection{Monomials and multi-indices}
All of the discussion up to this point has revolved around polynomials in $n^2$ variables, which quickly gets unwieldy unless one 
uses some better notation. To that end, 
\begin{defn}
	An $(n,r)$-\textbf{multi-index} $i$ is an $r$-tuple $(i_1,\dots,i_r)$ where each $i_j\in\underline n\eqdef\{1,\dots,n\}$.
	The collection of all $(n,r)$-multi-indices is denoted $I(n,r)$.
\end{defn}
\begin{rmk}
	One can also think of an element $i\in I(n,r)$ as a (set) map 
	\[i:\underline r\to\underline n.\]
\end{rmk}
The idea here is to associate to each monomial in a polynomial ring in many variables a tuple indicating its multidegree. That is we think of 
\[(i_1,\dots,i_r)\quad\leftrightsquigarrow\quad x_{i_1}\cdots x{i_r}\]
as corresponding to the same object. Which is wonderful except for one small flaw: polynomials are commutative 
and multi-indices (as we have defined them) aren't! For example, in $I(3,4)$,
\[(2,2,1,3)\quad\leftrightsquigarrow\quad x_1x_2^2x_3\quad\leftrightsquigarrow\quad (3,2,1,2).\]

To handle this disparity, we define an equivalence relation on $I(n,r)$ where we say that $i\sim j$ if they are in the 
same orbit under the natural $\frakS_r$ action. That is, if there exists $\sigma\in\frakS_r$ such that
\[(i_1,\dots,i_r)=(j_{\sigma(1)},\dots,j_{\sigma(r)})\]

In the context of polynomial representations of $\Gamma$, we want to consider polynomials in the coordinate functions $c_{ij}$,
so as a matter of notation if $i,j\in I(n,r)$, let $c_{i,j}$ denote the monomial 
\[c_{i,j}=c_{i_1j_1}\cdots c_{i_rj_r}.\]
Again, we want to take into account that we can permute the order on the right hand side, but now we need that $i_k$ and $j_k$ 
remain linked to the same function. To deal with this, we define an equivalence relation $\sim$ on $I(n,r)\times I(n,r)$ such that 
\[(a,b)\sim (c,d)\]
if there exists a $\sigma\in\frakS_r$ such that 
\[(a_1,\dots,a_r)=(b_{\sigma(1)},\dots,b_{\sigma(r)})\quad\text{and}\quad(c_1,\dots,c_r)=(d_{\sigma(1)},\dots,d_{\sigma(r)}).\]
The upshot of this work is that it gives us a bijection between (total) degree $r$ monomials in the $c_{ij}$ and the set
\[I(n,r)\times I(n,r)/\sim\]

\subsubsection{\texorpdfstring{$A_k(n,r)$}{Ak(n,r)}}
Notice that if $V\in M(n,r)$, each of its structure maps are homogeneous degree $r$ polynomials. As the first object of study, consider 
\begin{defn}
	Let $A_k(n,r)=A(n.r)$ denote the collection of all homogeneous degree $r$ polynomials in the 
	coordinate functions $c_{ij}:\Gamma\to k$.
\end{defn}
It is not too hard to see that 
\begin{prop}
	$A_k(n,r)$ is spanned by the elements 
	\[\{c_{i,j}|(i,j)\in I(n,r)\times I(n,r)\}\]
\end{prop}
however it takes a short argument to see 
\begin{lem}
	The dimension of $A_k(n,r)$ over $k$ is $\binom{n^2+r-1}{n^2-1}=\binom{n^2+r-1}{r}$.
\end{lem}
\begin{prf}
	The following is a ``stars and bars'' argument that is pervasive in combinatorics. See for example \cite{stanley} if unfamiliar with these techniques.
	
	Fix an ordering of the $c_{ij}$ (say the dictionary order)
	and relabel them $\{\gamma_1,\dots,\gamma_{m}\}$ (here $m=n^2$) according to this order. Then the degree $r$ monomials are in bijection with $m$-tuples $(a_1,\dots,a_{m})\in\bbN^m$ such that $\sum_i a_i=r$ via the map which sends 
	\[(a_1,\dots,a_{m})\mapsto \gamma_1^{a_1}\cdots\gamma_{m}^{a_{m}}.\]

	But choosing such an element is the same as inserting $m-1$ bars into a line of $r$ stars (that is an ordered partition of $r$ into $m$ parts, 
	where parts are allowed to be zero). But this is equivalent to choosing $m-1$ bars in a field of $m+r-1$ symbols. This is just 
	\[\binom{m+r-1}{m-1}\]
	and a well-known identity for binomial coefficients gets us the final equality.
\end{prf}
\begin{ex}
	In case the reader is unfamiliar with this kind of reasoning, consider the case when $n=5$ and $r=4$. Then the partition $(1,0,0,2,1)$ corresponding to 
	$\gamma_1\gamma_4^2\gamma_5$ corresponds to the stars-and-bars diagram 
	\begin{center}
		$\ast|||\ast\ast|\ast$
	\end{center}
	where there are $m+r-1=8$ symbols, $r=4$ of which are stars.
\end{ex}

\subsubsection{A dip into affine group schemes and category theory}
$A(n,r)$ lies within $k^\Gamma=k(\Gamma)$, the $k$-algebra of functions $\Gamma\to k$, which has the structure of a Hopf algebra induced from the group 
structure on $\Gamma$. More precisely, the functor $\GL_n:\Alg_k\to \Grp$ that assigns to every $k$-algebra $A$ the group $\GL_n(A)$ is representable. In other words, 
\[\GL_n(-)\simeq \Hom_{\Alg_k}(R,-)\]
where $R\cong k[x_{ij}|1\le i,j\le n]_{\det}$.

The anti-equivalence of the categories of affine group schemes over $k$ and finite dimensional commutative $k$-Hopf algebras follows from Yoneda lemma 
(c.f. \cite[chp. 1]{waterhouse}), and along with this equivalence comes a way to translate the group structure on $\Gamma$ 
into a coalgebra structure on $R$: we have maps $\mu,\epsilon$, the multiplication and unit maps on $\Gamma$ satisfying the diagrams 
\begin{center}
	\begin{tikzcd}
		\Gamma\times\Gamma\times\Gamma\ar[r,"\mu\times\id"]\ar[d,"\id\times\mu"] & \Gamma\times\Gamma\ar[d,"\mu"]\\
		\Gamma\times\Gamma\ar[r,"\mu"] & \Gamma
	\end{tikzcd}
	\quad \begin{tikzcd}
		\ast\times G\ar[r,"\epsilon\times\id"] &G\times G\ar[d,"\mu"]& G\times \ast\ar[l,"\id\times\epsilon",swap]\\
		& G\ar[ur,leftrightarrow,"\sim"]\ar[swap,ul,leftrightarrow,"\sim"] &
	\end{tikzcd}
\end{center}
(where $\ast$ is the trivial group) giving us associativity and identity. Yoneda gives us that the maps between functors (group schemes!)
\[\mu:\Gamma\times\Gamma\to \Gamma\quad\text{and}\quad \epsilon:\ast\to\Gamma\]
give rise to maps in $\Alg_k$:
\[\Delta\eqdef\mu^\ast:R\to R\otimes_k R\quad\text{and}\quad \varepsilon\eqdef\epsilon^\ast: R\to k\]
satisfying diagrams 
\begin{center}
	\begin{tikzcd}
		R\otimes R\otimes R & R\otimes R\ar[l,"\Delta\otimes\id",swap]\\
		R\otimes R\ar[u,"\id\otimes \Delta"] & R\ar[u,"\Delta"]\ar[l,"\Delta"]
	\end{tikzcd}
	\quad\begin{tikzcd}
		k\otimes R\ar[dr,"\sim",leftrightarrow,swap] & R\otimes R\ar[l,"\varepsilon\otimes \id",swap]\ar[r,"\id\otimes\varepsilon"] & R\otimes k\ar[dl,"\sim",leftrightarrow]\\
		& R\ar[u,"\Delta"] &
	\end{tikzcd}
\end{center}
\begin{prop}
	The maps $\Delta$ and $\varepsilon$ give $R$ a coalgebra structure. In coordinates, if $1\le i,j\le n$,
	\[\Delta(c_{ij})=\sum_k c_{ik}\otimes c_{kj}\quad\text{and}\quad \varepsilon(c_{ij})=\delta_{ij}\]
\end{prop}
In fact, as mentioned before, $R$ becomes a bialgebra (a Hopf algebra even, although we won't need the antipode here). This means that 
$\Delta$ and $\varepsilon$ are algebra morphisms for the natural algebra structure given by multiplication $m$ on $R$. In diagrams:
\begin{center}
	\begin{tikzcd}
		R^{\otimes 4}\ar[r,"\id\otimes\tau\otimes 1"] & R^{\otimes4}\ar[r,"m\otimes m"]  & R\otimes R\\
		R\otimes R\ar[u,"\Delta\otimes \Delta"]\ar[rr,"m"] & & R\ar[u,"\Delta"]
	\end{tikzcd}
	\quad\begin{tikzcd}
		R\otimes R\ar[r,"m"]\ar[d,"\varepsilon\otimes\varepsilon"] & R\ar[d,"\varepsilon"]\\
		k\otimes k\ar[r,"m"] & k
	\end{tikzcd}
\end{center}
where $\tau:R\otimes R\to R\otimes R$ is the twist map $a\otimes b\mapsto b\otimes a$. 
Chasing an element through the diagram on the left, we get
\[\tilde m\circ (\Delta\otimes \Delta)(c_{ij}\otimes c_{ab})=\sum_{1\le k,l\le n}c_{ik}c_{al}\otimes c_{kj}c_{lb}=\Delta(c_{ij}{c_{ab}})\]
or using our multi-index notation,
\[\Delta(c_{(i,a),(j,b)})=\sum_{(k,l)\in I(n,2)}c_{(i,a),(k,l)}\otimes c_{(k,l),(j,b)}.\]

Written more simply, the fact that $\Delta$ is an algebra morphism can be written 
\[\Delta(a\cdot b)=\Delta(a)\ast\Delta(b)\]
under suitable definitions of $\cdot$ and $\ast$. In a way that can be made precise, this means in particular that 
\[\Delta(a\cdot b\cdot c)=\Delta(a)\ast\Delta(b\cdot c)=\Delta(a)\ast\Delta(b)\ast\Delta(c)\]
and so on (since multiplication everywhere is associative) and therefore we can define this for arbitrary monomials and extend $k$-linearly: 
\begin{prop}
	If $i,j\in A(n,r)$, then 
	\[\Delta(c_{i,j})=\sum_{k\in I(n,r)}c_{i,k}\otimes c_{k,j}\quad\text{and}\quad \varepsilon(c_{i.j})=\delta_{i,j}\]
\end{prop}
One can easily see that degree is preserved by $\Delta$, meaning that 
\begin{prop}
	$\Delta$ and $\varepsilon$ descend to a coalgebra structure on $A(n,r)$. That is, $A(n,r)$ is a ($k$-)coalgebra.
\end{prop}

\subsubsection{The Schur Algebra}
Finally we get to the actual object of study:
\begin{defn}
	A \textbf{Schur algebra} is an element of the two-parameter family $\{S(n,r)\}=\{S_k(n,r)\}$ where $n$ and $r$ are any positive integers.
	As a set, $S(n,r)$ is the linear dual of $A(n,r)$:
	\[S(n,r)=A(n,r)^\ast=\Hom_k(A(n,r),k)\] 

	Let $\xi_{i,j}$ denote the element dual to $c_{i.j}\in A(n,r)$. In other words:
	\[\xi_{(a,b)}(c_{i,j})=\begin{cases}
		1, & (a,b)\sim(i,j)\\
		0, & \text{otherwise}
	\end{cases}\]
\end{defn}

\begin{lem}
	The coalgebra structure $(\Delta,\varepsilon)$ on $A(n,r)$ define an algebra structure on $S(n,r)$.
\end{lem}
\begin{prf}
	Since $k$ is an initial object in $\Alg_k$, there is a unique map $u:k\hookrightarrow S(n,r)$ sending $1$ to the constant one function $\1$, which is our unit map.
	Define multiplication $(\cdot)$ in $S(n,r)$ as follows: if $f,g\in S(n,r)$ then for any $x\in A(n,r)$ define 
	\[(f\cdot g)(x)=m_k\circ (f\otimes g)\circ \Delta(x)=\sum f(x_{(1)})g(x_{(2)})\]
	where $m_k:k\otimes k\to k$ denotes multiplication in $k$ and $\Delta(x)=\sum x_{(1)}\otimes x_{(2)}$ in Sweedler notation.

	Then we must just confirm that these maps satisfy the properties of a $k$-algebra. $(\cdot)$ is $k$-bilinear because (for instance)
	\begin{align*}
		((af+bg)\cdot h)(x)&=\sum (af+bg)(x_{(1)})\otimes h(x_{(2)})\\
		&=\sum a(f(x_{(1)})\otimes h(x_{(2)}))+b(g(x_{(1)})\otimes h(x_{(2)}))\\
		&= a\sum f(x_{(1)})\otimes h(x_{(2)})+ b\sum g(x_{(1)})\otimes h(x_{(2)})\\
		&=(a(f\cdot h)+b(g\cdot h))(x).
	\end{align*}
	
	It suffices to show that the unit $\1$ acts as it should on the spanning set $\xi_{i,j}$ for a basis element $c_{a,b}$:
	\[(\1\cdot \xi_{i,j})(c_{a,b})=\sum_{k=1}^n 1\cdot \xi_{i,j}(c_{k,b})\]
\end{prf}

\subsection{Representations of \texorpdfstring{$\frakS_n$}{Sn}}

\subsection{Explicit Examples}
To demonstrate the theory developed above, we begin a computation (in a simple case) of the isomorphism classes of irreducible 
representations of both $S_\bbC(2,2)$ and $\frakS_2$.

\subsubsection{The symmetric group on two letters}
The representation theory (over $k=\bbC$) of $\frakS_2$ is as simple as it comes: of course $\frakS_2\cong \bbZ/2\bbZ$ and we know that 
there are $|G|$ nonisomorphic irreducible representations of an abelian group $G$ over $\bbC$. Since we are talking about a symmetric group, 
we can realize these as the trivial and sign representations, represented by the Young diagrams:
\[\ytableausetup{smalltableaux,centertableaux}\ydiagram{2}\quad\text{and}\quad\ydiagram{1,1}\]

As submodules of the regular representation $k\frakS_2= k e\oplus k(1\,2)$, we can construct these as $\langle e+(1\, 2)\rangle$ (trivial representation) and $\langle e-(1\,2)\rangle$ (sign representation).

\subsubsection{The Schur algebra \texorpdfstring{$S_\bbC(2,2)$}{S(2,2)}}
Since $\ch\bbC=0$, (by e.g. \cite{schur-thesis}) we know immediately that $S_\bbC(2,2)$ is semisimple, so it suffices to identify the irreducible submodules therein.
Now we know 
\[S=S_\bbC(2,2)\cong \bbC^2\otimes\bbC^2\]
so $\dim_\bbC S=4.$ The theory outlined above give us that isomorphism types of irreducible modules are in bijection with partitions of 2 into two parts, meaning 
we have two isomorphism types: one corresponding to $\lambda_1=(1,1)$ and one corresponding to $\lambda_2=(2,1)$. 

Using the construction of $D_{\lambda,\bbC}$ from above, we can compute these two irreducible modules explicitly:

\begin{ex}[$\mathbf{\lambda_1=(1,1)}$]
 In this case our shape is $(1,1)$, corresponding to the Young diagram 
\[\ydiagram{1,1}\]
and then $D_{\lambda_1,\bbC}$ is spanned by the element
\[(T_l:T_{(2,1)})=\left(\ytableaushort{1,2}:\ytableaushort{2,1}\right)=c_{12}c_{21}-c_{11}c_{22}=c_{(1,2),(2,1)}-c_{(1,2),(1,2)}\in A_\bbC(2,2)\]
since all other bideterminants of this shape are zero or linearly dependent. Thus this is a one-dimensional irreducible representation.
\end{ex}
\begin{ex}[$\mathbf{\lambda_2=(2,0)}$]
Now our shape is $(2,0)$, corresponding to the diagram
\[\ydiagram{2}.\]
The bideterminants here are 
\begin{align*}
	(T_l:T_{(1,1)})=\big(\ytableaushort{1 1}:\ytableaushort{1 1}\big)=c_{11}^2\\
	(T_l:T_{(1,2)})=(T_l:T_{(2,1)})=c_{11}c_{12}\\
	(T_l:T_{(2,2)})=c_{12}^2
\end{align*}
So we have a three-dimensional irreducible representation spanned by $\langle c_{11}^2,c_{11}c_{12},c_{12}^2\rangle$.
\end{ex}
Since these are the only two Young diagrams of size two, these examples form a complete list of isomorphism classes of irreducible representations of $S_\bbC(2,2)$.

If we prefer instead to recognize our irreducibles as submodules of $E^{\otimes 2}$ (giving us a more obvious action by our algebras), 
we can use the short exact sequence 
\[0\to N\hookrightarrow E^{\otimes 2}\twoheadrightarrow D_{\lambda,\bbC}\to 0\]
to define the $n=\ker\pi$, where $\pi$ is defined to be the map in REFERENCE HERE. Then we can compute the orthogonal complement to $N$ to get $V_{\lambda,\bbC}$.
We can compute:
\[V_{\lambda_1,\bbC}=\langle e_1\otimes e_2-e_2\otimes e_1\rangle\]
and
\[V_{\lambda_2,\bbC}=\langle e_1\otimes e_1, \,e_1\otimes e_2+e_2\otimes e_1, \,e_2\otimes e_2\rangle\]
where $\{e_1,e_2\}$ is a basis for $E\cong k^2$.



%%%%%%%%%%%%%%%%%%%%%%%%%%%%%%%%%%%%%%%%%%%%%%%%%
%%%%%%%%%%%%%%%%%%%%%%%%%%%%%%%%%%%%%%%%%%%%%%%%%
%%%%%%%%%%%%%%%%%%%%%%%%%%%%%%%%%%%%%%%%%%%%%%%%%
%%%%%%%%%%%%%%%%%%%%%%%%%%%%%%%%%%%%%%%%%%%%%%%%%
%%%%%%%%%%%%%%%%%%%%%%%%%%%%%%%%%%%%%%%%%%%%%%%%%


\newpage
\section{The Schur-Weyl Functor}
From the discussion in the last section it is evident that the combinatorics behind the representation theory of $S(n,r)$ and $\frakS_r$ have some intersections
in their use of Young tableaux and this connection is more than superficial. In fact, there is a functor relating the representations
of these two objects in the following way:
\subsection{Construction of the functor \texorpdfstring{$\calF$}{F}}

Let $V\in M_k(n,r)$ be a $S(n,r)$-representation and select any weight $\alpha\in\Lambda(n,r)$. Then the weight space 
\[V^\alpha=\xi_\alpha V\]
becomes a $S(\alpha)\eqdef\xi_\alpha S(n,r)\xi_\alpha$-module using the action from $S(n,r)$. Now if we allow $r\le n$ and let
\[\omega=(1,\dots,1,0,\dots,0)\in\Lambda(n,r)\]
notice that $S(\omega)$ is spanned by the elements
\[\xi_\omega\xi_{i,j}\xi_\omega,\quad i,j\in I(n,r)\]
but by the multiplication rules established in the definition of $S(n,r)$, these are nonzero precisely when 
$i$ and $j$ are both of shape $\omega$. So then since $\xi_{i,j}=\xi_{i\sigma,j\sigma}$ for all $\sigma\in\frakS_r$, we can take as
a basis of $S(\omega)$ the set 
\[\{\xi_{u\pi,u}|\pi\in\frakS_r\}\]
where $u=(1,2,\cdots,r)\in I(n,r)$.

To prove the next statement we require a computational result.
\begin{lem}\label{lem:somega-mult}
	If $u=(1,2,\dots,r)\in I(n,r)$, then for all $\pi,\sigma\in\frakS_r$,
	\[\xi_{u\pi,u}\cdot \xi_{u\sigma,u}=\xi_{u\pi\sigma,u}.\]
\end{lem}
\begin{prf}
	Using the formulas for multiplication in $S(n,r)$, recall that 
	\begin{equation}
		\xi_{u\pi,u}\cdot\xi_{u\sigma,u}=\sum Z_{i,j} \xi_{i,j}\label{eq:1}
	\end{equation}
	where 
	\[Z_{i,j}=\#\{s\in I(n,r)|(u\pi,u)\sim(i,s)\text{ and }(u\sigma,u)\sim (s,j)\}.\]
	Then for each $i,j$, since $u=(1,2,\dots,r)$ has no stabilizer in $\frakS_r$, there is a unique 
	$g$ such that $u\pi g=i$, meaning that $s=ug$. 

	But then this fixes (again a unique) $h\in\frakS_r$ such that $u\sigma h=s=u g$ whence $\sigma h= g$. 
	One computes that 
	\[u\pi\sigma h = u\pi g=i\quad\text{and}\quad uh = j\]
	therefore since in the above computation $s$ was completely determined by $i$, we have
	\[Z_{i,j}=\left\{\begin{array}{lr}
		1, &  (i,j)\sim(u\pi\sigma,u)\\
		0, & \text{otherwise}
	\end{array}\right.\]
	and the result follows.
\end{prf}
Using this result, we prove a more obviously useful statement:

\begin{lem}
	$S(\omega)\cong k\frakS(r)$.
\end{lem}
\begin{prf}
	Define the map $\varphi:S(\omega)\to k\frakS_r$ on the basis above to be 
	\[\varphi (\xi_{u\pi,u})=\pi\]
	and extending $k$-linearly.

	This is a homomorphism since 
	\[\varphi(\xi_{u\pi,u}\xi_{u\sigma,u})=\varphi(\xi_{u\pi\sigma,u})=\pi\sigma=\varphi(\xi_{u\pi,u})\varphi(\xi_{u\sigma,u})\]
	and it is bijective since it is bijective on the respective bases and is thus bijective as a linear map.
\end{prf}
The upshot of these lemmas is that one can define the functor 
\[\calF:M_k(n,r)\to \Rep(\frakS_r)\]
via the map that sends any representation $V$ to its $\omega$ weight space $V^\omega\in \lmod {S(\omega)}\simeq \Rep(\frakS_r)$.


\subsection{Properties of \texorpdfstring{$\calF$}{F}}


%%%%%%%%%%%%%%%%%%%%%%%%%%%%%%%%%%%%%%%%%%%%%%%%%
%%%%%%%%%%%%%%%%%%%%%%%%%%%%%%%%%%%%%%%%%%%%%%%%%
%%%%%%%%%%%%%%%%%%%%%%%%%%%%%%%%%%%%%%%%%%%%%%%%%
%%%%%%%%%%%%%%%%%%%%%%%%%%%%%%%%%%%%%%%%%%%%%%%%%
%%%%%%%%%%%%%%%%%%%%%%%%%%%%%%%%%%%%%%%%%%%%%%%%%


\section{Strict Polynomial Functors}


%%%%%%%%%%%%%%%%%%%%%%%%%%%%%%%%%%%%%%%%%%%%%%%%%
%%%%%%%%%%%%%%%%%%%%%%%%%%%%%%%%%%%%%%%%%%%%%%%%%
%%%%%%%%%%%%%%%%%%%%%%%%%%%%%%%%%%%%%%%%%%%%%%%%%
%%%%%%%%%%%%%%%%%%%%%%%%%%%%%%%%%%%%%%%%%%%%%%%%%
%%%%%%%%%%%%%%%%%%%%%%%%%%%%%%%%%%%%%%%%%%%%%%%%%

\section{Questions and Extensions}

\section*{Acknowledgements}
\label{sec:ack}
\addcontentsline{toc}{section}{\nameref{sec:ack}}
I extend my most heartfelt thanks to my advisor, Julia Pevtsova, who not only helped me immensely in setting a target 
for this project, but also introduced me to many of the classical ideas found in this paper (some times more than once). 
Her knowledge and understanding while I learned this subject has been absolutely invaluable to me.

My thanks also to my loving partner, who stands besides me in good times and in bad and always patiently humors me when 
I need someone to spew math at (or a shoulder to cry on).

Finally, thank you to my friends and colleagues in the University of Washington math department for many fruitful conversations 
and for helping me to foster my natural mathematical curiosity into a raging fire.

%%%%%%%%%%%%%%%%%%%%%%%%%%%%%%%%%%%%%%
%%%%%%%%%%  Bibliography %%%%%%%%%%%%%
%%%%%%%%%%%%%%%%%%%%%%%%%%%%%%%%%%%%%%
\medskip

\printbibliography
\addcontentsline{toc}{section}{References}

\end{document}